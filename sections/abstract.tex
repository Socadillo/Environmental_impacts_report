
\setcounter{secnumdepth}{0}                    %% Removes section number

\section{Abstract}
Most European rivers are impacted by various anthropogenic pressures and in need for action in order to reach a good ecological status as required by the EU Water Framework Directive. Since benthic invertebrates are good bio-indicators, they are often used to monitor rivers and to assess organic pollution, habitat availability and overall degradation. For this study the multi-habitat sampling approach was carried out in the unimpacted river Ois and in the channelized Maiergraben in order to assess differences in benthic invertebrate communities. At the unimpacted river different mineral habitats, biotic cover and flow velocities are present, whereas the impacted site is very homogenous. Furthermore, a higher number of taxa, EPT-Taxa and sensitive taxa occur at the Ois, which is a result of high habitat heterogeneity. The Maiergraben on the other hand shows low taxa diversity and very high abundances of generalists such as chironomids. In order to reach a good ecological status at the Maiergraben the need for action should not be ignored but appropriate measures should be implemented to restore the rivers biodiversity.\footnote{Data not available from EUWWR.}

 