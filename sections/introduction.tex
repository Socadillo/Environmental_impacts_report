

\section{Introduction}\label{sec:introduction}                                                   %% The first section

Limited access to clean water has become an increasingly alarming concern around the world. This is not the current situation in Austria, where most of the rivers have low levels of organic pollution. However, through the pursuit of hydropower production and flood protection, Austria’s water bodies face other anthropogenic pressures such as channelization, impoundment and hydropeaking~\parenciteA{Muhar2000}. The EU Water Framework Directive 2000/60/EC (WFD) established the legal obligation of member countries to maintain healthy water resources and mitigate damages inflicted upon their water bodies. The WFD describes a water body’s status by both abiotic and biotic criteria. Biological quality elements are subdivided into assessment of benthic invertebrates, fish, algae and aquatic flora. The ecological status of a water body is defined by comparing the biological community composition present with the near-natural reference conditions. The ecological status is calculated from the residual between the observed condition and the reference condition~\parenciteA{Furhacker2008}.

Benthic invertebrate organisms live in a diverse range of habitats representing a wide variety of aquatic ecosystems. Community structure of Macrozoobenthos (MZB) responds to environmental disturbances in a rather predictable way. This makes them an excellent candidate for monitoring changes in environmental conditions~\parenciteA{Li2010}. MZBs are able to reflect different anthropogenic pressures through changes in structure or function of the assemblages. These changes allow for an overall assessment of streams. Aside from organic pollution, MZB can detect habitat loss and overall stream degradation~\parenciteA{Hering2004, Hering2006}. According to~\textcite{Marzin2012}, macroinvertebrate metrics appear to be more sensitive to the degradation of the overall condition of the river than fish metrics. This could be explained by the localized nature of MZB. The insect order trichoptera are particularly well suited as a bioindicators for describing habitat degradation~\parenciteA{Schmidt-Kloiber2017}.

Many benthic invertebrate species have adapted to specific habitat parameters by way of flow velocity preference, structural needs, feeding strategies and tolerance to pollution~\parenciteA{Stoll2016}. It has been shown that substrate size is one of the best predictors of benthic invertebrate distribution within a river by~\textcite{Jowett2003} and~\textcite{Schroder2013}; while~\textcite{Dohet2015} described the influence that thermal regime and land use has on MZB inhabiting headwater streams. A study of species diversity and functional feeding groups of benthic invertebrates in Austrian rivers, which was carried out by~\citeauthoryearpar{Yoshimura2006}, showed that MZB communities are dependent on their environmental conditions. Studies in Poland have shown that benthic communities have significantly reduced taxonomic richness in constrained channels~\parenciteA{Wyzga2014}.

\begin{figure}[!htb]                                                       %% Reference site photo
  \center
  \includegraphics[width=.95\linewidth]{images/site_reference}                 %% Width, Image file COLOR
  \caption{Reference sample location on Ois River.}                            %% Figure Caption
  \label{fig:site_reference}                                                        %% Figure label key
\end{figure}


This report focuses on the benthic invertebrate communities sampled from two Alpine rivers near Lunz-am-see. We collected and assessed a sample of the benthic community from a near-natural site of the Ois River~(\cref{fig:site_reference}). This was to be used as a reference site when compared to a sample which was collected from a heavily impacted site of the Maiergraben. By comparing the collected samples from the Ois and Maiergraben, we aim to answer the following questions:

\singlespacing                                              %% Set single spacing
\begin{enumerate}
  \item Does habitat availability differ between the unimpacted Ois River and the impacted Maiergraben?
  \item Is there a relation between choriotope and sensitive screening taxa?
  \item What influence do microhabitats have on benthic communities within a river?
  \item Is there a difference in taxa composition, diversity and abundance between the unimpacted Ois and impacted Maiergraben?
  \item Does the benthic community represent the observed impacts at the Maiergraben?
\end{enumerate}
\onehalfspacing                                             %% Sets 1.5 line spacing


We hypothesized that the hydro-morphologically dynamic Ois River will contain more diverse habitat than the constrained Maiergraben. We predicted that some screening taxon are associated with specific choriotopes. We also hypothesized that habitat heterogeneity will result in more taxon richness and abundance, with an increase in specialized taxa. We predicted that the taxa composition, diversity and abundance would differ between the two sites, with the Ois River having a higher diversity and abundance than the Maiergraben. We assumed that the benthic community of the Maiergraben will be comprised of taxon associated with heavily impacted and channelized rivers.

